\section{Entwicklungsumgebung}


Für das Entwickeln der Java Microservices braucht man Java 17 LTS, sowie Maven (mit einer Version größer gleich 3.8.5). Bevor man die Entwicklungsumgebung starten kann muss man die \textit{docker-compose.yml} Datei im Hauptordner des Projektes mittels \textit{docker-compose up} starten.

Sobald Docker den rabbitmq und monodb container gestartet hat, kann man mit dem starten der Microservices beginnen. 

An sich gibt es keine strikte Reihenfolge für das starten der Microservices, jedoch bietet es sich an folgenden Reihenfolge zu befolgen, um die Wartezeit zu verringern. 

\begin{enumerate}
	\item Entity-Service
	\item Tracking-Service
	\item Flocontrol-Service
	\item Simulator-Service
	\item Gateway-Service
	\item Cockpit
\end{enumerate}

\subsection{Starten der Java Services}

Die Services \textit{Flowcontrol-Service}, \textit{Simulato-Service} und \textit{Gateway-Service} können mittels Maven gestartet werden. Hierfür muss man in den Ordner des gewünschten Services gehen und den folgenden command ausführen:  \\ \verb|mvn spring-boot:run|

\subsection{Starten der Typescript Services}
JAN MAGIC

\subsection{Konfiguration der Services}
Einige Services der Applikation bieten Anpassungsmöglichkeit. Diese Anpassungen werden über Environment variablen gesteuert. Die Applikationen laufen auch ohne das Anpassen dieser Werte.

\subsubsection{Konfiguration Entity-Service}
Folgende Environment variablen sind möglich zu setzen.


\begin{table}[h]
	\begin{tabular}{|l|l|l|}
		\hline
		Name & Standardwert & Beschreibung \\ \hline
		\verb|MONGO_DB_HOST| & localhost:27017 &  Hostname / die ip wo die MongoDB Instanz läuft \\ \hline
		\verb|MONGO_DB_USER| & admin  &  Username für die Authentifizierung bei der MongoDB Instanz \\ \hline
		\verb|MONGO_DB_PWD| & admin &  Passwort für die Authentifizierung bei der MongoDB Instanz \\ \hline
		\verb|MONGO_DB_NAME| & local-entity-db &  MongoDB Datenbank die das Service benutzen soll \\ \hline
		\verb|RABBIT_MQ_HOST| & localhost &  Hostname / die IP wo die RabbitMQ Instanz läuft \\ \hline
	\end{tabular}
	\caption{Environment variablen für Entity-Service }
\end{table}

\subsubsection{Konfiguration Tracking-Service}
Folgende Environment variablen sind möglich zu setzen.


\begin{table}[h]
	\begin{tabular}{|l|l|l|}
		\hline
		Name & Standardwert & Beschreibung \\ \hline
		\verb|MONGO_DB_HOST| & localhost:27017 &  Hostname / die ip wo die MongoDB Instanz läuft \\ \hline
		\verb|MONGO_DB_USER| & admin  &  Username für die Authentifizierung bei der MongoDB Instanz \\ \hline
		\verb|MONGO_DB_PWD| & admin &  Passwort für die Authentifizierung bei der MongoDB Instanz \\ \hline
		\verb|MONGO_DB_NAME| & local-tracking-db &  MongoDB Datenbank die das Service benutzen soll \\ \hline
		\verb|RABBIT_MQ_HOST| & localhost &  Hostname / die IP wo die RabbitMQ Instanz läuft \\ \hline
	\end{tabular}
	\caption{Environment variablen für Tracking-Service }
\end{table}

\subsubsection{Konfiguration Simulator-Service}
Folgende Environment variablen sind möglich zu setzen.

\begin{table}[h]
	\begin{tabular}{|l|l|l|}
		\hline
		Name & Standardwert & Beschreibung \\ \hline
		\verb|ENTITY_SERVICE_IP| & localhost &  Hostname / die IP vom Entity-Service \\ \hline
		\verb|TRACKING_SERVICE_IP| & localhost  &  Hostname / die IP vom Tracking-Service  \\ \hline
		\verb|FLOW_SERVICE_IP| & localhost &  Hostname / die IP vom Flowcontrol-Service  \\ \hline
		\verb|SPRING_RABBITMQ_HOST| & localhost &  Hostname / die IP wo die RabbitMQ Instanz läuft  \\ \hline		
	\end{tabular}
	\caption{Environment variablen für Simulator-Service }
\end{table}

\subsubsection{Konfiguration Flowcontrol-Service}
Folgende Environment variablen sind möglich zu setzen.

\begin{table}[h]
	\begin{tabular}{|l|l|l|}
		\hline
		Name & Standardwert & Beschreibung \\ \hline
		\verb|ENTITY_SERVICE_IP| & localhost &  Hostname / die IP vom Entity-Service \\ \hline
		\verb|TRACKING_SERVICE_IP| & localhost  &  Hostname / die IP vom Tracking-Service  \\ \hline
		\verb|SIMULATOR_SERVICE_IP| & localhostdmin &  Hostname / die IP vom Simulator-Service  \\ \hline
		\verb|SPRING_RABBITMQ_HOST| & localhost &  Hostname / die IP wo die RabbitMQ Instanz läuft  \\ \hline		
	\end{tabular}
	\caption{Environment variablen für Simulator-Service }
\end{table}

\subsubsection{Konfiguration Gateway-Service}
Folgende Environment variablen sind möglich zu setzen.

\begin{table}[h]
	\begin{tabular}{|l|l|l|}
		\hline
		Name & Standardwert & Beschreibung \\ \hline
		\verb|ENTITY_SERVICE_IP| & localhost &  Hostname / die IP vom Entity-Service \\ \hline
		\verb|TRACKING_SERVICE_IP| & localhost  &  Hostname / die IP vom Tracking-Service  \\ \hline
		\verb|SIMULATOR_SERVICE_IP| & localhostdmin &  Hostname / die IP vom Simulator-Service  \\ \hline
		\verb|FLOW_SERVICE_IP| & localhost &  Hostname / die IP vom Flowcontrol-Service  \\ \hline
	\end{tabular}
	\caption{Environment variablen für Gateway-Service }
\end{table}

