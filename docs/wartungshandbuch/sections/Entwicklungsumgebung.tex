\section{Entwicklungsumgebung}

Für das Entwickeln der Java Microservices werden Java 17, Maven 3.8.5 sowie Yarn 1.22.18 oder neuer benötigt.
Bevor man die Entwicklungsumgebung starten kann, müssen eine Datenbank sowie eine \enquote{MoM} mithilfe von Docker Compose gestartet werden.
Dazu kann der Befehl \verb|docker-compose up -d| im Hauptverzeichnis des Projektes ausgeführt werden.

Es gibt keine strikte Reihenfolge für das Starten der Microservices.
Die folgende Reihenfolge bietet sich jedoch an, um die Wartezeit zu verringern. 

\begin{enumerate}
	\item Entity-Service
	\item Tracking-Service
	\item Flowcontrol-Service
	\item Simulator-Service
	\item Gateway-Service
	\item Cockpit
\end{enumerate}

\subsection{Starten der Java Services}

Die Services \textit{Flowcontrol-Service}, \textit{Simulator-Service} und \textit{Gateway-Service} können mittels Maven gestartet werden.
Hierfür muss im Verzeichnis des gewünschten Dienstes der Befehl \verb|mvn spring-boot:run| ausgeführt werden.

\subsection{Starten der Typescript Services}

Die TypeScript Services sowie das Cockpit lassen sich über den Befehl \verb|yarn dev| im jeweiligen Verzeichnis starten.
Zuvor muss der Befehl \verb|yarn install| verwendet werden, um benötigte Bibliotheken zu installieren. 


\subsection{Konfiguration der Services}

Einige Services der Applikation bieten Anpassungsmöglichkeiten über Umgebungsvariablen, wobei Standardwerte eine Konfiguration optional machen.

\subsubsection{Konfiguration Entity-Service}

Folgende Umgebungsvariablen sind anpassbar.


\begin{table}[h]
	\begin{tabular}{|l|l|l|}
		\hline
		Name & Standardwert & Beschreibung \\ \hline
		\verb|MONGO_DB_HOST| & localhost:27017 & Adresse der MongoDB Instanz \\ \hline
		\verb|MONGO_DB_USER| & admin & Username der MongoDB Instanz \\ \hline
		\verb|MONGO_DB_PWD| & admin & Passwort der MongoDB Instanz \\ \hline
		\verb|MONGO_DB_NAME| & local-entity-db & Name der MongoDB Datenbank \\ \hline
		\verb|RABBIT_MQ_HOST| & localhost & Adresse RabbitMQ Instanz \\ \hline
	\end{tabular}
	\caption{Environment variablen für Entity-Service }
\end{table}

\subsubsection{Konfiguration Tracking-Service}

Folgende Umgebungsvariablen sind anpassbar.


\begin{table}[h]
	\begin{tabular}{|l|l|l|}
		\hline
		Name & Standardwert & Beschreibung \\ \hline
		\verb|MONGO_DB_HOST| & localhost:27017 & Adresse der MongoDB Instanz \\ \hline
		\verb|MONGO_DB_USER| & admin & Username der MongoDB Instanz \\ \hline
		\verb|MONGO_DB_PWD| & admin & Passwort der MongoDB Instanz \\ \hline
		\verb|MONGO_DB_NAME| & local-entity-db & Name der MongoDB Datenbank \\ \hline
		\verb|RABBIT_MQ_HOST| & localhost & Adresse RabbitMQ Instanz \\ \hline
	\end{tabular}
	\caption{Environment variablen für Tracking-Service }
\end{table}

\subsubsection{Konfiguration Simulator-Service}

Folgende Umgebungsvariablen sind anpassbar.

\begin{table}[h]
	\begin{tabular}{|l|l|l|}
		\hline
		Name & Standardwert & Beschreibung \\ \hline
		\verb|ENTITY_SERVICE_IP| & localhost & Adresse des Entity-Service \\ \hline
		\verb|TRACKING_SERVICE_IP| & localhost & Adresse des Tracking-Service \\ \hline
		\verb|FLOW_SERVICE_IP| & localhost & Adresse des Flowcontrol-Service \\ \hline
		\verb|SPRING_RABBITMQ_HOST| & localhost & Adresse der RabbitMQ Instanz \\ \hline		
	\end{tabular}
	\caption{Environment variablen für Simulator-Service }
\end{table}

\subsubsection{Konfiguration Flowcontrol-Service}

Folgende Umgebungsvariablen sind anpassbar.

\begin{table}[h]
	\begin{tabular}{|l|l|l|}
		\hline
		Name & Standardwert & Beschreibung \\ \hline
		\verb|ENTITY_SERVICE_IP| & localhost & Adresse des Entity-Service \\ \hline
		\verb|TRACKING_SERVICE_IP| & localhost & Adresse des Tracking-Service \\ \hline
		\verb|SIMULATOR_SERVICE_IP| & localhost & Adresse des Simulator-Service \\ \hline
		\verb|SPRING_RABBITMQ_HOST| & localhost &  Adresse der RabbitMQ Instanz \\ \hline		
	\end{tabular}
	\caption{Environment variablen für Simulator-Service }
\end{table}

\subsubsection{Konfiguration Gateway-Service}

Folgende Umgebungsvariablen sind anpassbar.

\begin{table}[h]
	\begin{tabular}{|l|l|l|}
		\hline
		Name & Standardwert & Beschreibung \\ \hline
		\verb|ENTITY_SERVICE_IP| & localhost & Adresse des Entity-Service \\ \hline
		\verb|TRACKING_SERVICE_IP| & localhost & Adresse des Tracking-Service \\ \hline
		\verb|SIMULATOR_SERVICE_IP| & localhost & Adresse des Simulator-Service \\ \hline
		\verb|FLOW_SERVICE_IP| & localhost & Adresse des Flowcontrol-Service \\ \hline
	\end{tabular}
	\caption{Environment variablen für Gateway-Service }
\end{table}

