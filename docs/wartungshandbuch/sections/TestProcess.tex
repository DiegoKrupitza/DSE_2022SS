\section{Test Process}

Für jeden Microservice wurden mindestens zwei Modultests implementiert.

Die Java-basierten Services wurden mithilfe von \textit{JUnit5} getestet.
Hierbei wurde die Springboot Version \verb|2.6.5| verwendet.

Die Tests der TypeScript-basierten Services nutzen den \textit{Test Runner Vitest}  mit Version \verb|0.12.6|.

\subsection{Gateway}

Beim Gateway wird getestet, ob alle Services und deren Status vom Gateway richtig angezeigt werden.
Die \textit{Health-Check} Anfragen an die jeweiligen Services werden mithilfe von \textit{Mockito} gemockt. 

Die Tests können mit dem Befehl \textit{mvn clean test} in der Kommandozeile ausgeführt werden.

\subsection{Flowcontrol-Service}

Die Tests des Flowcontrol-Service überprüfen das Berechnen von optimalen Geschwindigkeit für Fahrzeuges in der nähe von Ampeln.
Das Ziel der Berechnung ist, dass Autos so selten wie möglich zu einem totalen Stillstand kommen.

Der Flowcontrol-Service kommuniziert in der Produktionsversion mit einer RabbitMQ Instanz.
Damit die Tests unabhängig von RabbitMQ laufen können, kommt hier ein Testcontainer mit dem Image \textit{rabbitmq:3-management} zum Einsatz.
Zusätzlich wird die Kommunikation mit anderen Microservices mithilfe von \textit{Mockito} gemockt.

Die Tests können mit dem Befehl \textit{mvn clean test} in der Kommandozeile ausgeführt werden. 

\subsection{Simulator-Service}

Das korrekte Verhalten beim Starten und Stoppen einer Simulation wird mit den vorhandenen Integrationstests überprüft.
Dabei werden sowohl eingehende Anfrage an den Service als auch ausgehende Anfragen an weiteren Services mithilfe von \textit{Mockito} gemockt.

Die Tests können mit dem Befehl \textit{mvn clean test} in der Kommandozeile ausgeführt werden. 

\subsection{Entity-Service}

Die vorhandenen Integrationstests überprüfen, ob die Endpunkte den Anforderungen entsprechend verhalten und der Service seinen Status über seinen \textit{health}-Endpunkt korrekt übermittelt.
Die Anfragen werden mithilfe eines im Test erstellten Testservers verarbeitet.
Zusätzlich wird eine \textit{In-Memory-MongoDB} Instanz verwendet.

Die Tests können mit dem Befehl \textit{yarn test} in der Kommandozeile ausgeführt werden. 

\subsection{Tracking-Service}

Die vorhandenen Integrationstests überprüfen, ob die Endpunkte den Anforderungen entsprechend verhalten und der Service seinen Status über seinen \textit{health}-Endpunkt korrekt übermittelt.
Die Anfragen werden mithilfe eines im Test erstellten Testservers verarbeitet.
Zusätzlich wird eine \textit{In-Memory-MongoDB} Instanz verwendet.

Die Tests können mit dem Befehl \textit{yarn test} in der Kommandozeile ausgeführt werden.
