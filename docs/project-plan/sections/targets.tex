\section{Projektzielplan}

Ziel dieses Projektes ist die Entwicklung eines Softwaresystems, welches eine Kommunikation zwischen autonomen Fahrzeugen und Verkehrsinfrastruktur simuliert.
Dabei soll eine intelligente Verkehrflusssteuerung auf Basis eines Algorithmus umgesetzt werden.
Die Umsetzung soll auf Basis eines verteilten Systems erfolgen, wobei einzelne Komponenten voneinander entkoppelt implementiert werden sollen.
Dafür sollen Komponenten mithilfe von REST und einer \enquote{MoM} kommunizieren.

Folgende Projektziele werden realisiert:

\begin{itemize}
  \item Es ist möglich eine Verkehrsituation mit bis zu vier Fahrzeugen und vier Ampeln zu simulieren.
  \item Einzelne Komponenten sind voneinander entkoppelt und sind daher horizontal skalierbar.
  \item Eine Web-Applikation erlaubt Benutzerinnen und Benutzern das Bedienen des Systems über einen Browser.
  \item Benutzerinnen und Benutzern ist es möglich die Anzahl an Fahrzeugen und Ampeln einer Simulation sowie deren Startzustände zu konfigurieren.
  \item Für die Geschwindigkeitsberechnung eines Fahrzeuges berücksichtigt der Algorithmus nur Ampeln, in deren Scanreichweite sich das betroffene Fahrzeug befindet.
  \item Der Algorithmus arbeitet mit realistischen Geschwindigkeiten in dem Bereich von 0 bis 130 km/h. Ein Stillstand (0 km/h) erfolgt nur, wenn ein Fahrzeug vor einer roten Ampel halten muss.
  \item Die Simulation ist so vereinfacht, dass nur eine Koordinate relevant ist.
\end{itemize}

Folgende Projektziele werden nicht realisiert:

\begin{itemize}
  \item Der implementierte Algorithmus entspricht einer optimalen Lösung in Bezug auf Zeit- und Speicherkomplexität.
  \item Komponenten verwenden Autorisierung und Authentifizierung um Kommunikation abzusichern.
  \item Der Algorithmus findet die global optimale Lösung, indem auch Ampeln außerhalb der Scanreichweite berücksichtigt werden.
  \item Das System simuliert realistische Beschleunigungs- und Bremsvorgänge.
  \item Die Web-Applikation ist für Mobilgeräte optimiert.
  \item Das System verfügt über Ausfallsicherheit.
  \item Das Service Mesh wird visualisiert.
  \item Eine gelbe Ampelphase wird bei der Simulation berücksichtigt.
\end{itemize}
