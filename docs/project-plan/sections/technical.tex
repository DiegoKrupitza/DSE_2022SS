\section{Technische Planung}

Ein Eureka stellt neben einer Registrierung für einzelne Dienste sowie darauf aufbauender Service Discovery auch Statusüberwachung bereit.

Als \enquote{MoM} für eine asynchrone Kommunikation wird RabbitMQ verwendet.

Ein Spring Cloud Gateway wird als Einstiegspunkt für externe Kommunikation dienen.
Dafür werden Anfragen an zuständige Komponenten weitergeleitet und Load-Balancing durchgeführt.

Zum Persistieren von Daten wird MongoDB verwendet, da die Datenbestände keine Relationen zwischen Entitäten aufweisen und sie sich somit für dokumentbasierte Datenbanken eignen.

EntityService und TrackingService werden mit Node.js, TypeScript sowie Fastify implementiert.

FlowControlService und SimulatorService werden hingegen mit Spring Boot entwickelt.
Dabei steht insbesondere die simple Konfiguration im Vordergrund.

Die Web-Applikation wird mit Vue 3 und Vite entwickelt, um den Anteil an Boilerplate zu minimieren und schnelles Iterieren während der Entwicklung zu ermöglichen.

Während der Entwicklung werden Docker Compose und minikube verwendet um das System lokal in Betrieb zu nehmen.
