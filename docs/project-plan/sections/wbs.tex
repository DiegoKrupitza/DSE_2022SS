\section{Work Breakdown Structure (WBS)}

Die folgende Aufzählung enthält alle Arbeitspakete.
Die Zahl hinter dem Titel eines Arbeitspaketes entspricht einer Aufwandsschätzung in Story Points (SP).

\begin{enumerate}
  \item Aufsetzen eines Netflix Eureka Servers (1 SP)\\
  		In diesem Arbeitspaket wird eine Netflix Eureka Server Instanz aufgesetzt, damit diese in späterer Folge für Service Registrierung sowie Discovery verwendet werden kann.
  \item Aufsetzen eines RabbitMQ Servers (1 SP)\\
  		In diesem Arbeitspaket wird mithilfe von Docker Compose eine RabbitMQ Instanz aufgesetzt, welche als \enquote{MoM} dient.
  \item Aufsetzen einer MongoDB Instanz (1 SP)\\
  		In diesem Arbeitspaket wird eine MongoDB Instanz für die Entwicklungsumgebung aufgesetzt.
  \item Konfigurieren eines Spring Cloud Gateways (2 SP)\\
  		In diesem Arbeitspaket wird ein Gateway implementiert, welcher Anfragen der Web-Applikation an die zugehörigen Dienste weiterleitet.
  		Beim Starten des Dienstes soll eine Registrierung beim Eureka Server erfolgen.
  \item EntityService - Aufsetzen (2 SP)\\
  		In diesem Arbeitspaket wird die Grundstruktur des EntityServices auf Basis von TypeScript implementiert damit in späterer Folge darauf aufgebaut werden kann.
  		Das EntityService soll sich nach dem Starten beim Eureka Server registrieren.
  		Es soll ein Dockerfile zum Bauen eines Docker Images erstellt werden.
  \item EntityService - Implementieren eines Endpunktes zur Registrierung von Fahrzeugen (2 SP)\\
  		In diesem Arbeitspaket soll das Empfangen und Persistieren von relevanten Fahrzeuginformationen auf Basis der \enquote{MoM} implementiert werden.
  		Dies soll über die \enquote{MoM} geschehen.
  \item EntityService - Registrierung von Ampeln (2 SP)\\
  		In diesem Arbeitspaket wird eine Queue \enquote{traffic-light} von RabbitMQ gelesen und bei eintreffenden Nachrichten die Datenbank aktualisiert.
  \item EntityService - Implementieren eines Endpunktes zum Lesen von Datensätzen (2 SP)\\
  		In diesem Arbeitspaket sollen REST-Endpunkte implementiert werden, welche das Lesen von Ampel- und Fahrzeuginformationen ermöglicht.
  		Es soll insbesondere möglich sein Daten einzelner Entitäten über deren IDs zu lesen.
  \item EntityService- Implementieren eines Endpunktes zum Durchführen von geospatial Queries zu Ampelpositionen (2 SP)\\
  		In diesem Arbeitspaket sollen REST-Endpunkte implementiert werden, welche die Ergebnisse von geospatial Queries zu den Standorten von Ampeln liefern.
  \item EntityService - Zurücksetzen (1 SP)\\
 		In diesem Arbeitspaket wird ein REST-Endpunkt implementiert, welcher es ermöglicht den gesamten Zustand des Dienstes zurückzusetzen.
  \item TrackingService - Aufsetzen (2 SP)\\
  		In diesem Arbeitspaket wird die Grundstruktur des TrackingService auf Basis von TypeScript implementiert damit in späterer Folge darauf aufgebaut werden kann.
  		Der TrackingService soll sich nach dem Starten beim Eureka Server registrieren.
  		Es soll ein Dockerfile zum Bauen eines Docker Images erstellt werden.
  \item TrackingService - Anbindung an "MoM" für Ampeln (2 SP)\\
  		In diesem Arbeitspaket wird eine Queue \enquote{traffic-light-state} von RabbitMQ gelesen und bei eintreffenden Nachrichten die Datenbank aktualisiert.
  		Ampeln senden neben ihrem Status zusätzlich die verbleibende Zeit bis zur nächsten Statusänderung.
  \item TrackingService - Anbindung an "MoM" für Fahrzeuge (2 SP)\\
  		In diesem Arbeitspaket wird eine Queue \enquote{car-state} von RabbitMQ gelesen und bei eintreffenden Nachrichten die Datenbank aktualisiert.
  		Ein Fahrzeug sendet Standort, Fahrtrichtung, sowie seine aktuelle Geschwindigkeit.
  \item TrackingService - Lesen von Daten (2 SP)\\
  		In diesem Arbeitspaket werden REST-Endpunkte implementiert, welcher das Lesen aller Fahrzeugpositionen sowie Ampelzustände ermöglicht.
  		Insbesondere soll es möglich sein alle Zustände sowie den aktuellsten Zustand einer Entität über deren ID zu lesen.
  \item TrackingService - Zurücksetzen (1 SP)\\
 		In diesem Arbeitspaket wird ein REST-Endpunkt implementiert, welcher es ermöglicht den gesamten Zustand des Dienstes zurückzusetzen.
 		Dafür sollen alle persistierten Daten gelöscht werden.
  \item SimulatorService (Endpoint) - Aufsetzen (3 SP)\\
  		In diesem Arbeitspaket wird der SimulatorService aufgesetzt.
  		Bei Startup soll sich der Dienst sich bei Eureka registrieren.
  		Neben einem Worker, welcher für die Verwaltung der Simulationsentitäten zuständig ist, soll es einen REST-Endpunkte und \enquote{MoM} Anbindungen zur externen Kommunikation geben.
  \item SimulatorService (Endpoint) - Simulation von Ampeln (2 SP)\\
  		In diesem Arbeitspaket wird die Simulation von Ampeln implementiert.
  		Ampeln wechseln periodisch zwischen den Zuständen \enquote{rot} und \enquote{grün} und senden bei jedem Wechsel ihren aktuellen Zustand an die \enquote{MoM} Queue \enquote{traffic-light-state}.
  \item SimulatorService (Endpoint) - Simulation von Fahrzeugen (3 SP)\\
  		In diesem Arbeitspaket wird die Simulation von Fahrzeugen implementiert.
  		Fahrzeuge fahren mit einer gegebenen Geschwindigkeit und melden periodisch ihren aktuellen Zustand an die \enquote{MoM} Queue \enquote{car-state}.
  		Zudem wird die Queue \enquote{car-speed} auf eingehenden Nachrichten beobachtet und bei Bedarf die Geschwindigkeit von Fahrzeugen angepasst.
  \item SimulatorService (Endpoint) - Szenario erstellen (2 SP)\\
  		In diesem Arbeitspaket wird ein REST-Endpunkt zum Erstellen und Starten von Szenarios implementiert.
  		Dieser soll alle relevanten Informationen empfangen und darauf aufbauend Simulationen für Fahrzeuge und Ampeln starten.
  \item SimulatorService (Endpoint) - Zurücksetzen (2 SP)\\
  		In diesem Arbeitspaket wird ein REST-Endpunkt implementiert, welcher beim Aufruf die aktuelle Simulation terminiert.
  \item SimulatorService (Endpoint) - Flux Kompensator (2 SP)\\
  		In diesem Arbeitspaket wird ein Zeitraffer für Simulationen implementiert.
  		Der Faktor einer Zeitraffung soll beim Erstellen eines Szenarios anpassbar sein.
    \item FlowControlService - Aufsetzen (2 SP)\\
  		In diesem Arbeitspaket wird die Grundstruktur des Spring Boot FlowControlService implementiert damit in späterer Folge darauf aufgebaut werden kann.
  		Der FlowControlService soll sich nach dem Starten beim Eureka Server registrieren.
  		Es soll ein Dockerfile zum Bauen eines Docker Images erstellt werden.
  \item FlowControlService - Algorithmus (5 SP)\\
  		In diesem Arbeitspaket soll der Algorithmus zur Optimierung des Verkehrsflusses implementiert werden.
  		Dazu soll der FlowControlService bei der \enquote{MoM} auf Nachrichten einer Queue \enquote{car-state} achten, welche Standortaktualisierungen von Fahrzeugen melden.
  		Bei Eintreffen solcher Nachrichten soll die optimale Geschwindigkeiten betroffener Fahrzeuge ermittelt werden, sodass diese noch bei \enquote{grün} die Ampel passieren können beziehungsweise, falls möglich, vor der Ampel nicht zum Stillstand kommen müssen.
  		Die so ermittelten Geschwindigkeiten sollen über die \enquote{MoM} Queue \enquote{car-speed} an die betroffenen Fahrzeuge gesendet werden.
  		Es muss eine zulässige Höchstgeschwindigkeit implementiert werden.
  \item Cockpit - Aufsetzen (1 SP)\\
  		In diesem Arbeitspaket wird die grundlegende Einrichtung der Web-Applikation durchgeführt.
  		Es soll ein Dockerfile zum Bauen eines Docker Images erstellt werden.
  \item Frontend - Simulationen starten (1 SP)\\
 		In diesem Arbeitspaket wird ein Formular zum Starten von Simulationen mit konfigurierbaren Parametern implementiert.
  \item Frontend - Simulationen visualisieren (3 SP)\\
		In diesem Arbeitspaket wird eine Visualisierung von laufenden Simulationen in der Web-Applikation implementiert.
		Es sollen die Positionen von Fahrzeugen und Ampeln in einer schematischen Karte angezeigt werden
  \item Frontend - Informationen von Entitäten anzeigen (2 SP)\\
  		In diesem Arbeitspaket wird eine Anzeige von Informationen über Fahrzeuge und Ampeln implementiert.
  \item Swagger Dokumentation (3 SP)\\
  		In diesem Arbeitspaket wird der API Gateway mittels Swagger dokumentiert.
  \item Deployment auf der GCP (5 SP)\\
  		In diesem Arbeitspaket soll das Deployment, welches während der Entwicklung auf Basis von Docker Compose und Minikube stattfindet, auf die Google Cloud Platform migriert werden.
\end{enumerate}
