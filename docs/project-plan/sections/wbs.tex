\section{Work Breakdown Structure (WBS)}

Die folgende Aufzählung enthält alle Arbeitspakete.
Die Zahl hinter dem Titel eines Arbeitspaketes entspricht einer Aufwandschätzung in Story Points (SP).

\begin{enumerate}
  \item Aufsetzen eines Netflix Eureka Servers (1 SP)\\
  		In diesem Arbeitspaket wird eine Netflix Eureka Server Instanz aufgesetzt, damit diese in späterer Folge für Service Registrierung sowie Discovery verwendet werden kann.
  \item Aufsetzen eines RabbitMQ Servers (1 SP)\\
  		In diesem Arbeitspaket wird mithilfe von Docker Compose eine RabbitMQ Instanz aufgesetzt, welche als \enquote{MoM} dient.
  \item Aufsetzen einer MongoDB Instanz (1 SP)\\
  		In diesem Arbeitspaket wird eine MongoDB Instanz für die Entwicklungsumgebung aufgesetzt.
  \item Konfigurieren eines Spring Cloud Gateways (2 SP)\\
  		In diesem Arbeitspaket wird ein Gateway implementiert, welcher Anfragen der Web-Applikation an die zugehörigen Dienste weiterleitet.
  		Beim Starten des Dienstes soll eine Registrierung beim Eureka Server erfolgen.
  \item Aufsetzen des EntityService (2 SP)\\
  		In diesem Arbeitspaket wird die Grundstruktur des EntityServices auf Basis von TypeScript implementiert damit in späterer Folge darauf aufgebaut werden kann.
  		Das EntityService soll sich nach dem Starten beim Eureka Server registrieren.
  		Es soll ein Dockerfile zum Bauen eines Docker Images erstellt werden.
  \item EntityService - Implementieren eines Endpunktes zur Registrierung von Fahrzeugen (2 SP)\\
  		In diesem Arbeitspaket soll das Empfangen und Persistieren von relevanten Fahrzeuginformationen auf Basis der \enquote{MoM} implementiert werden.
  		Dies soll über die \enquote{MoM} geschehen.
  \item EntityService - Registrierung von Ampeln (2 SP)\\
  		In diesem Arbeitspaket soll das Empfangen und Persistieren von relevanten Ampelinformationen auf Basis der \enquote{MoM} implementiert werden.
  \item EntityService - Implementieren eines Endpunktes zum Lesen von Entitätsinformationen (2 SP)\\
  		In diesem Arbeitspaket sollen REST-Endpunkte implementiert werden, welche das Lesen von Ampel- und Fahrzeuginformationen ermöglicht.
  		Es soll insbesondere möglich sein Daten einzelner Entitäten über deren IDs zu lesen.
  \item EntityService- Implementieren eines Endpunktes zum Durchführen von geospatial Queries zu Ampelpositionen (2 SP)\\
  		In diesem Arbeitspaket sollen REST-Endpunkte implementiert werden, welche die Ergebnisse von geospatial Queries zu den Standorten von Ampeln liefern.
  \item 
\end{enumerate}

