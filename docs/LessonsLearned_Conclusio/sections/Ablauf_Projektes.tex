\section{Beurteilung des Projektablaufs}

\subsection{Was hat gut funktioniert, was weniger gut? Begründen Sie ihre Beurteilung.}

Im Großen und Ganzen war der Ablauf während der Entwicklung sehr gut.
Aufgrund einer optimalen Projektkommunikation, dank regelmäßiger Meetings und Statusupdate innerhalb der Projektchats, haben wir sehr effizient arbeiten können.
Des Weiteren wurden mithilfe von Pair-Programming Sessions Probleme technischer Natur schnell und effizient gelöst.

Der Start des Projektes verlief ein wenig holprig, da aufgrund von Abweichungen bei den Entwicklungsumgebungen der Entwickler Probleme bei der Inbetriebnahme der Services entstanden.
Diese konnten jedoch durch rasche Kommunikation behoben werden.

Sehr gut hat der Einsatz eines zentrales Repositories bei GitHub funktioniert, in welchem alle Artefakte der Projektes hinterlegt wurden.
Dies ermöglichte allen Entwickler jederzeit einen Zugriff auf den aktuellsten Stand des Projektes, ohne auf eine Vielzahl and Diensten zugreifen zu müssen.
Insbesondere profitierte davon die Versionierung der Dokumentationsdokumente, welche mit \LaTeX \space erstellt wurden.

\subsection{Wenn zutreffend, warum haben Sie den ursprünglich geschätzten Aufwand überschritten?}

Die Aufwandsschätzung war relative genau und konnte dank einem Einsatz von modernen Frameworks zum Teil unterboten werden.

Hingegen wurde beim Einrichten des Deployments auf GKE beziehungsweise eines lokalen Kubernetes Clusters viel Zeit verloren, da das Team bisher kaum Erfahrung mit diesen Technologien gesammelt hatte und anfangs widersprüchliche Tutorials hinzugezogen wurden.

Letztendlich konnte dieser Mehraufwand durch eine effiziente Entwicklung ausgeglichen werden.

\subsection{Was würden Sie aufgrund der gewonnenen Erfahrung anders bzw. besser machen?}

Das Deployment der Anwendung wurde zu Beginn der Entwicklung nicht beachtet.
Daher hatten wir einen Netflix-Eureka-Service für Service-Discovery konfiguriert und eingebunden.
Erst beim Deployment ist uns aufgefallen, dass GKE diese Funktionalität nativ bereitstellt.

Deshalb mussten wir unsere Service-Discovery-Logik sowie deren Integration im Gateway umschreiben.
Aufgrund dieser gewonnenen Erfahrung, werden wir zukünftig bereits zu Beginn der Entwicklung das Deployment sowie die Produktionsumgebung berücksichtigen. 
