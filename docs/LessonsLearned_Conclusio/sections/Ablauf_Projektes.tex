\section{Beurteilung Ablauf des Projektes}

\subsection{Was hat gut funktioniert, was weniger gut? Begründen Sie ihre Beurteilung.}

Im Großen und Ganzen war der Ablauf während der Entwicklung sehr gut. Aufgrund einer optimalen Projektkommunikation, dank regelmäßiger Meetings und Statusupdate innerhalb der Projektchats, haben wir sehr effizient arbeiten können. Des Weiteren wurden aufgrund Pair-Programming Sessions, Probleme Technischer Natur sehr schnell aus der Welt geschafft und somit wurde nicht viel Zeit verloren. Der Start des Projektes war ein wenig holprig da aufgrund von Versions unterschieden (insbesondere Java) nach fehlern gesucht werden musste. Was sehr gut funktioniert hatte, war das wir ein Zentrales Gitrepository erstellt hatten in dem alle Artefakte der Projektes hinterlegt wurden. Dies ermöglichte uns jederzeit den neuesten Stand zu sehen ohne auf viele verschiedene Services zugreifen zu müssen. Bei der Dokumentation war dies nur möglich, da wir mit \LaTeX gearbeitet haben.

\subsection{Wenn zutreffend, warum haben Sie den ursprünglich geschätzten Aufwand überschritten?}

Die Aufwandsschätzung war relative genau und wir haben sogar aufgrund von Frameworks uns einiges an zeit gespart, obwohl wir bei einigen uns zuerst in die Dokumentation einlesen mussten. Hingegen bei dem Setup vom Deployment auf GKE bzw der Lokalen Kubernetes Engine haben wir recht viel Zeit verloren, da bisher kaum Erfahrung damit hatten und anfangs widersprüchliche Tutorials geschaut und Dokumentationen gelesen haben. Jedoch dieser Mehraufwand wurde aufgrund der effizienten Entwicklung ausgeglichen.

\subsection{Was würden Sie aufgrund der gewonnenen Erfahrung anders bzw. besser machen?}

Die Hauptaufgabe des Gateways ist bei uns hauptsächlich das Weiterleiten von Anfragen. Nächstes mal würden wir versuchen, den Gateway selbst schon Anfragen bearbeiten zu lassen, damit das Frontend nicht so viele einzelne Requests schicken muss, nur Eine, die dann das Gateway direkt bearbeitet.  
Am Anfang des Projetkes wurde das Deployment nicht beachtet. Daher hatten wir für Service-Discovery ein Netflix-Eureka Service genutzt. Beim Deployment ist uns dann aufgefallen, dass GKE dies für uns übernimmt. Dadurch mussten wir unsere Service-Discovery-Logik und deren Integration im Gateway umschreiben. Aufgrund dieser gewonnenen Erfahrung ist es empfehlenswert, bereits am Beginn der Entwicklung, das Deployment und Production-Environment zu beachten. 
